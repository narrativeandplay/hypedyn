\documentclass{article}
\usepackage{graphicx}
\usepackage{hyperref}

\begin{document}

\title{The HypeDyn Hypertext Fiction Editor\\Tutorial 4: Number Facts}
% \author{Alex Mitchell}
\date{}

\onecolumn
\maketitle

\tableofcontents

\section{Introduction}
In this tutorial, we will introduce ``number facts'', which are similar to text
facts and true\slash false facts but can be used to store \textit{numbers}.
Number facts can be updated to a specific number, to the value stored in another
number fact, or to a random number. You can also perform simple math on number
facts, and use number facts in conditions. We will be creating a variation of
the ``Little Red Riding Hood'' hypertext fiction which you created in tutorial
1. We will change the story such that the choice of ending is determined by a
number fact, which can either be preset by the author or chosen randomly when
the story is being read. We will also change the side path (``Hood details'')
such that it is unlocked, not after the first reading, but instead after the
\textit{second} reading, through the use of a counter.

% number facts:
% create number fact
% update number fact using: input, fact, math, random
% update text using: number fact
% conditions: 
% - examples of order of execution, using stop-if-true, intermediate fact for
% complex calculation, several rules working together in one link/node rule
% - explain about what is triggered when (in sidebar?)
% random? - random endings: woodsman may or may not show up to save Red

The nodes and facts in the final story are shown in Figure
\ref{fig:tut3:completed}.

\begin{figure}[h]
  \centering
  \includegraphics[width=11cm]{images/hypedyn-tutorial-4-figure-1}
  \caption{\textit{The completed ``Little Red Riding Hood'' story.}}
  \label{fig:tut3:completed}
\end{figure} 

\textit{Note:  HypeDyn is a work-in-progress, so there are some features that are still
not completed, and there may be bugs. If you encounter any errors, please
report them as bugs on our Launchpad site: \url{https://launchpad.net/hypedyn}.}

\section{Getting started}

First, open HypeDyn by double-clicking on the file \textbf{HypeDyn.exe} (in
Windows) or \textbf{HypeDyn.app} (in MacOS).

We will continue from the story that you created in tutorial 1. If you don't
have the story, you can start from the file \textsc{LRRH.dyn} in the
\textsc{examples} folder. HypeDyn files always end with a \textbf{.dyn} extension.
Once you've opened \textsc{LRRH.dyn}, save the file under a different 
name, such as \textsc{LRRH5.dyn}.

\section{Creating a number fact}

Number facts are similar to the text facts and true/false facts which 
were introduced in Tutorial 3. Instead of keeping track of text or 
boolean values, number facts are \textit{integers}, meaning they can be set 
to any positive or negative ``natural'' or ``whole'' number ie. \dots -3, 
-2, -1, 0, 1, 2, 3 \dots, but not \textit{real} numbers with a 
decimal place. Number facts can be tested in conditions, and set in 
actions.

The first feature we will introduce is the use of a number fact in a condition.
As with other conditions, this allows you to determine which actions should be
triggered, either when a node is entered or when a link is clicked.

\subsection{Creating the fact}
% - create a number fact: whichEnding?

We will begin by creating the number fact.

\begin{enumerate}
  \item In the main HypeDyn window, go to the \textit{Fact} menu, and 
  pick the \textit{New} menu item, and the \textit{Number} submenu item. 
  \item Name the fact ``whichEnding?'', and click "Ok". As with other 
  facts, your new fact will appear in the fact list in the lower left 
  corner of the main window.  
\end{enumerate}

We now have a \textit{number fact} which we can use to keep track of 
numeric values.

\subsection{Creating the alternative endings}
% - make two endings: version 1 and version 2

What we want to do with our "whichEnding?" number fact is use it to 
determine which of two possible endings will be visited when the 
reader clicks on ``next'' in the ``forest'' node. Currently it leads 
to the node ``end''. We will now create 2 new nodes, ``Version 1'' 
and ``Version 2'', which will be shown depending on the value of 
``whichEnding?''. Create two nodes:

% should continue the numbering here, do this later
\begin{enumerate}
  \item \textit{Name}: Version 1\\
  \textit{Content}: 
  \begin{quotation}
  \noindent Quickly the wolf ran ahead to Grandma's house, swallowed 
  Grandma whole, and disguised himself as the poor old lady. When Red 
  arrived, he finished her off too. Yum! \\

  \noindent next
  \end{quotation}
  \item \textit{Name}: Version 2\\
  \textit{Content}: 
  \begin{quotation}
  \noindent Quickly the wolf ran ahead to Grandma's house. 
  Unfortunately for the wolf, Grandma's friend the Woodcutter had 
  stopped by for tea, and he finished the wolf off with one swipe of 
  his axe. Poor wolf! \\

  \noindent next
  \end{quotation}
  \item Add a link from the ``next'' text in each node to the ``end'' node.
\end{enumerate}


\subsection{Setting the number fact}
% - set number fact in node rule for ``forest''

We want to have the reader go from the ``next'' link to ``Version 1'' 
if ``whichEnding?'' is equal to 1, and to ``Version 2'' if 
``whichEnding?'' is equal to 2. Before we can do this, however, we 
need to set the value of ``whichEnding?'' As with other types of 
facts, a number fact can be updated in either a node rule or a link 
rule. We will use the node rule in the ``forest'' node to set the 
value.       

\begin{enumerate}
    \item Edit the ``forest'' node, and edit its node rule.
    \item Add a rule, and name the rule ``choose ending''.
    \item Edit the rule, and add an ``update fact'' action.
    \item In the new action, choose ``Number'' for the fact type, and 
    choose fact ``whichEnding?'' for the fact.
    \item Notice that there are several options in the ``using'' 
    pulldown menu. Number facts can be updated based on a number 
    that the author inputs directly (the \textit{input} option), much 
    like specifying alternative text when updating text on a link. 
    They can also be updated using another fact, using a mathematical 
    expression, or using a random number
    
    For now, we want to specify the number directly, so choose 
    ``input''. In the input field, specify ``1''.
\end{enumerate}

\noindent The rule we've created will set the ``whichEnding?'' fact to 1 when 
the reader enters the ``forest'' node. 

\subsection{Displaying a number fact using alternative text}

To check whether our new rule is 
working, we can temporarily add a link which we will use to display 
the current value of ``whichEnding?'' This technique is a useful way 
to debug your stories when using number facts.

\begin{enumerate}
    \item At the end of the ``forest'' node, add the text 
    ``whichEnding? =  
    dummy''. We will replace the text ``dummy'' with the current 
    value of whichEnding?
    \item Create a link on the text ``dummy'', name the link 
    ``debugging'', and edit the link.
    \item Add a rule to the link, and name the rule ``debugging''.
    \item Add an ``update text using'' action to the rule, and choose 
    update text using ``number fact''. In the pulldown menu, choose 
    the fact ``whichEnding?''
    \item Close the rule editor and the link editor.
    \item Now run the story. Notice that the text ``dummy'' is 
    replaced with the value ``1''.
    \item Edit the ``forest'' node's node rules, go into the ``choose 
    ending'' rule, and change the ``update fact using'' action so 
    that the value is 2. Close the rule and node rules editors.
    \item Run the story again. You should see that the value 
    displayed in the forest node is now ``2''.
\end{enumerate}

\noindent Now that we've created our number fact, and have set it to a specific 
value, we can use this number fact to decide which ending to show.

\section{Comparing number facts}

% - now make use of this in a condition
We will now add a condition to the ``to the end'' link in the 
``forest'' node, which checks the value of the ``whichEnding?'' fact. 
If the value is 1, the link will go to ``Version 1'', and if the 
value is 2, the link will go to ``Version 2''.

\subsection{Adding the first rule}

We'll begin by adding the rule which takes us to ``Version 1''.

\begin{enumerate}
    \item Edit the ``forest'' node, and then edit the ``to the end'' 
    link. You should see the ``to the end'' rule which we created in 
    tutorial 1.
    \item Edit the ``to the end'' rule. It currently contains one 
    action, ``follow link to end''. Rename the rule ``to version 1''.
    \item Now change the action to ``follow link to Version 1''. This 
    is the action we will use if ``whichEnding?'' is 1.
    \item Add a condition. Choose ``Number Fact'' as the type of 
    condition.
    \item Notice that after changing the condition to a number fact 
    condition, the condition changes to show two pulldown menus and a 
    text input field. The first pulldown menu contains =, $<$, $>$, 
    $\le$ and $\ge$. This is the \textit{comparator} 
    which will be used to compare the chosen number fact (in this 
    case ``whichEnding?'') with the right-hand side of the condition.
    
    For now, leave the comparator as ``=''.
    \item To the right of the comparator is another pulldown menu, 
    which has two options: input and fact. Choosing ``Input'' lets 
    you enter a specific number for the comparison, whereas choosing 
    ``Fact'' lets you compare against another fact. Leave the choice 
    as ``Input''.
    \item In the text entry field, enter ``1''. This means that the 
    actions listed in the rule will be triggered when 
    ``whichEnding?'' is equal to 1.
    \item Close the rule editor.
\end{enumerate}

\subsection{Testing the first rule}

At this point, we've created the rule which will take the reader to 
node ``Version 1'' if the fact ``whichEnding?'' is equal to 1. Recall 
that we have currently set the node rule in the ``forest'' node to 
set ``whichEnding?'' to 2. What will happen if we run the story now?

\begin{enumerate}
    \item Run the story.
    \item Notice that the link on the text ``next'' in ``forest'' 
    is disabled -- this is because it contains rule with a ``follow link to'' 
    action, but that rule's conditions are not satisfied.
    \item Now edit the ``choose ending'' node rule in the ``forest'' 
    node, and change the ``update fact'' action so that 
    ``whichEnding?'' is set to 1.
    \item Run the story again. This time, the link is enabled. Click 
    on the link, and you will be taken to node ``Version 1''.
\end{enumerate}

\subsection{Adding the second rule}

% - create condition that checks if 1 or 2 (stop if true used)

Now we need to add the second rule, which will take the reader to 
node ``Version 2'' if ``whichEnding?'' is 2.

\begin{enumerate}
    \item Edit the ``to the end'' link in the ``forest'' node.
    \item We are going to add a second rule, which should only ever 
    be triggered if the first rule, ``to version 1'', was 
    \textbf{not} satisfied. To make sure that HypeDyn stops checking 
    rules after a given rule is satisfied, you can check the ``Stop 
    if true'' option on a rule. Do this now.
    \item Add a rule, and name the rule ``to version 2''.
    \item Add an action ``follow link to Version 2''.
    \item We only want this rule to be triggered if ``whichEnding?'' 
    is set to 2, so we need to add a condition to check this. 

    % stop if true discussion - why put the stop if true if we're 
    % also putting a condition? for efficiency
    
    Add a condition to the rule, and set it to ``Number fact 
    whichEnding? = input 2''.
    \item Run the story. When you get to ``forest'' and click on the 
    ``next'' link, you should go to ``Version 1''.
    \item Now edit the node rule for ``forest'', and change the 
    ``choose ending'' rule so that ``whichEnding?'' is set to 2.
    \item Run the story again. This time, the ``next'' link should 
    take you to ``Version 2''.
\end{enumerate}

\noindent We have now successfully created a link which goes to different 
destinations based on a number fact.

%One thing to note is that both the ``Stop if true'' option on the 
%``to version 1'' rule and the condition in the ``to version 2'' rule 
%will ensure that the second rule is only triggered if 

% Note: not sure how useful this next bit is
% - could use a fact rather than typing in the number
% - create fact preferredEnding
% - set to 1 in start node, then set whichEnding to fact preferredEnding

\section{Using random numbers}

% - but this isn't very interesting - why not make it random?
Although we've created a set of rules which use number facts to 
determine which node to visit next, this isn't really anything which 
we couldn't have done with, for example, a true\slash false fact. One way 
to make this more interesting, and to show what can be done with 
number facts, is to set the fact ``whichEnding?'' to a \textit{random 
number}. 

\subsection{Setting the random number in a node rule}

We will now create a rule which sets ``whichEnding?'' to a random 
number. There are several places where we could put this rule. To 
start with, we will change the ``choose ending'' node rule in the 
``forest'' node to update ``whichEnding?'' to a random number, rather 
than a number which we decided when writing the story.

% edit node rule, instead of setting to fact, set to random between input 1
% and input 2
\begin{enumerate}
    \item Edit the ``choose ending'' rule in the node rules for the 
    ``forest'' node.
    \item There is currently one action: ``update fact Number 
    whichEnding? using Input 2''. Change the ``using'' pulldown to 
    ``random''. Notice that there are now two input fields to the 
    right of ``random''. These input fields allow you 
    to choose the lower and upper range of the random number which 
    will used to update the number fact. These bounds can be 
    specified either using Input or a Fact. For now, leave both 
    pulldowns as Input.
    \item We want a random number between 1 and 2, since there are 2 
    versions of the ending. So enter ``1'' in the first input field, 
    and ``2'' in the second input field. 
    \item Click ``Ok'' to close the rule editor, and ``Close'' to 
    close the node rules editor. Now run the story. When you enter 
    the ``forest'' node, the value displayed at the bottom of the 
    node should be either 1 or 2.
    \item To convince yourself that the value is being chosen 
    randomly, click on the ``back'' button, and then click on the 
    ``forest'' link again. Do this a few times. Notice that sometimes 
    the same number appears two or more times in a row -- this is 
    because the number is chosen randomly, without regard to what the 
    previous value was. 
    % aside: if you wanted it to alternate, how would you do this?
\end{enumerate}

\subsection{Setting the random number in a link rule}

% or could do it when the link is clicked - useful to show order of 
% evaluation
Note that although we put the update fact action which sets 
``whichEnding?'' on the node rule for the ``forest'' node, instead we 
could have put the action on the ``next'' link itself. One advantage 
of having the action on the node rule is we can see the value of the 
``whichEnding?'' fact (on the ``dummy'' link in the node's text), 
which helps with debugging. However, it can sometimes be useful to 
put actions such as this on a link. To show how this is done, we will 
now put the same action that we put in the node rule into a rule on 
the ``next'' link.

\begin{enumerate}
    \item Edit the ``next'' link in the ``forest'' node.
    \item There are already two rules on the link: ``to version 1'' 
    and ``to version 2''. We want to update the ``whichEnding?'' 
    fact, and \textit{then} check the value of the fact. Remember 
    that HypeDyn checks, and then triggers, rules in the order they 
    are listed in the rules editor. This means that we will have to 
    put our new rule before the two existing rules.
    
    Add a new rule. Notice that it appears at the end of the list of 
    rules. We need to move it up so that it is before ``to version 
    1''.
    \item Make sure your new rule is selected (it should be 
    highlighted in blue''. Now click on the ``up'' button. The rule 
    should have moved up, so that it is now before ``to version 2'', 
    but still after ``to version 1''.
    \item Click ``up'' again. The new rule should now be first in the 
    list, before ``to version 1''.
    \item Edit the new rule, and name it ``randomize''.
    \item Add an ``update fact'' action, set it to ``update fact 
    Number using Random between Input 1 and Input 2''.
    \item Click ``Ok'' to close the rule editor, and then click 
    ``Close'' to close the link rules editor.
    \item Try running the story, and clicking the ``next'' link 
    several times. Notice that the destination the link goes to is 
    not always the same as what you would expect based on the number 
    displayed in the ``forest'' node. This is because the ``next'' 
    link's first action is again randomizing the value in 
    ``whichEnding?''
\end{enumerate}

\subsection{Adding a third ending}

% - what if we add a third ending? need to change both randomizations
% - instead, make a ``numberOfEndings'' fact, set in start node, use this
% instead (replace above update from fact with this?)

Note that it is not very good form to have two different places where 
``whichEnding?'' is randomized, as you may lose track of where the 
randomization is taking place. There is also another problem -- if we 
add another ending, we now need to remember to change the upper bound 
for the random number from two to three in two different places!

Sometimes there are good reasons for having randomization such as 
this happening in several places. For example, there may be several 
different nodes which lead to the endings, each of which need to do 
some randomization based on the number of endings. One way to improve 
the situation is to use a \textit{fact} for the upper bound, rather 
than a number we have typed in two different places. Then we can just 
change the value of the fact, and everywhere the fact is used, we'll 
get the correct value. We will do that now.

\begin{enumerate}
    \item First, we'll make a third version of the ending. Create a 
    new node, and name it ``Version 3''.
    \item Enter the following text in the node:
    
    \begin{quotation}
    \noindent Fortunately for Red, the wolf got lost on the way to 
    Grandma's house.\\

    \noindent next
    \end{quotation}
    \item Add a link from the text ``next'' to the ``end'' node.
    \item Now we will add a new number fact to keep track of how many 
    endings there are. Create a new number fact, and name it 
    ``numberOfEndings''.
    \item We need to update this fact to ``3''. The best place to do 
    this is at the start of the story. 
    
    Edit the ``start'' node, and edit the node rules for this node.
    \item Add a new rule, and call the rule ``initialize''. This is 
    where we will set the value of ``numberOfEndings''.
    \item Add an ``update fact'' action to the rule, and set it as 
    follows: ``update fact Number numberOfEndings using Input 3''.
    \item Now we need to make use of this new fact in our two rules 
    where we randomize ``whichEnding?''.
    
    Edit the ``forest'' node, and edit its node rules. Edit the rule 
    ``choose ending''. In the first action, change the second 
    ``Input'' to ``Fact''. This is the upper bound of our random 
    number. Choose the fact ``numberOfEndings''.
    \item Now do the same for the ``randomize'' rule in the ``to the 
    end'' link's rules.
    \item Finally, we need to add a rule to the ``to the end'' link 
    to take the reader to node ``Version 3'' if ``whichEnding?'' is 
    equal to 3.
    
    Add a rule to the ``to the end'' link's rules, and name it ``to 
    version 3''. Make sure the rule is at the end of the list of 
    rules.
    \item Add the condition ``Number Fact whichEnding? = Input 3''.
    \item Add the action ``follow link to Version 3''.
\end{enumerate}

\noindent Now try running the story. You should be able to follow the ``next'' 
link in the ``forest'' node, and have it randomly take you to one of 
the three endings.

In this section, we introduced the use of a random number fact to 
create a link which randomly takes the reader to different 
destinations. This is a powerful procedural technique. It can, 
however, be problematic if misused. If choices the reader is making 
result in random outcomes, the reader may begin to doubt that her 
choices have any real impact on the story. This may or may not be 
what you intend as an author.

\section{Using a counter}

% - counter: unlock side path on third reading
In this section, we will introduce another way in which authors can 
use number facts to make their stories more procedural. Here, we will 
update facts based on math. We will demonstrate this by implementing 
a \textit{counter} which unlocks a side path in the story after the 
reader has reread the story a certain number of times. To do this, we 
will create a new number fact, named ``counter''. Each time the 
reader clicks on the ``restart'' link from the ``end'' node back to 
the ``start'' node, we will add 1 to ``counter''. Then, in the ``more 
about the hood'' link, instead of checking whether or not the ``end'' 
node has been visited, we will check the value of ``counter'', and 
enable the link when the target value has been reached.

\subsection{Creating the counter}

First we will create a number fact to act as our counter.

% - add ``counter'' number fact
\begin{enumerate}
    \item Create a new number fact, and name it ``counter''. We'll 
    use this to keep track of the number of times the reader has gone 
    back to the start of the story from the ``end'' node.
    \item So that we can see how the counter is changing, we will use 
    a technique similar to what we did with the ``whichEnding?'' fact 
    to display the value of ``counter''.
    
    Edit the ``end'' node, and add the text ``counter=placeholder'' 
    between the text ``*** The End ***'' and the link ``back to 
    start''. Add a link on the text ``placeholder'', name the link 
    ``show counter'', and add a rule which has the action ``update 
    text using number fact counter''. This will show us the value of 
    ``counter'' when we test our story.
\item Now run the story. When you get to the ``end'' node, notice that the 
value shown for ``counter'' is ``0''. 
\end{enumerate}

\noindent Number facts are always 
automatically updated to zero when the story is started. In this case 
this is what we want, as the reader has initially not gone back to 
the start of the story.

\subsection{Incrementing the counter}

 When the reader goes back to the start the 
first time, we will add 1 to the current value of counter, updating 
counter to 1.

% - add ``update number fact counter using math fact counter + input 1'' to
% ``restart'' link in ``end''
\begin{enumerate}
    \item Edit the ``restart'' link in the ``end'' node. 
    \item There is already one rule, ``restart'', which was added in 
    Tutorial 1. This rule contains one action, ``follow link to 
    start'', which takes the reader back to the ``start'' node. In 
    addition, we want to increment the counter fact when the reader 
    goes back to the ``start'' node.
    
    Add an ``update fact'' action to the ``restart'' rule. Set the 
    type of update to ``Number'', select Fact ``counter'', and set 
    the ``using'' pulldown menu to ``Math''.
    \item After you choose ``Math'' for the ``using'' pulldown menu, 
    you will see an ``Input'' pulldown menu, a ``+'' pulldown menu, 
    and a second ``Input'' pulldown menu. These menus let you specify 
    the \textit{math expression} which will be triggered for this 
    action. 
    
    HypeDyn lets you update number facts using simple math 
    expressions. Each one consists of two values (either an Input or 
    a Fact), and an \textit{operator}: +, -, x, / (div) and \% (mod). 
    Because number facts are whole numbers, the ``/'' operator is 
    actually ``div'', which performs the division but discards the 
    remainder. The ``\%'' operator is called ``mod'', which performs 
    the division and gives you the remainder.
    
    We want to add 1 to the current value of ``counter''. To do this, 
    choose ``Fact'' for the first value. You will see the text field 
    is replaced by a pulldown menu containing a list of number facts. 
    Choose ``counter''.
    \item For the second value, we want to add 1, so leave the second 
    value's pulldown menu as ``Input'', and enter ``1'' into the text 
    field. 
    \item Finally, we want to perform addition on the two values, so 
    leave the operator pulldown showing ``+''.
    \item Click ``Ok'', and then close the link rules editor.
\end{enumerate}

Try running the story now. Click through to the ``end'' node, and 
then read the story a second time by following the ``back to start'' 
link. When you reach the ``end'' node a second time, you should see 
the text ``counter=1''. Try rereading several times, and you'll see 
the counter increase by 1 each time.

\subsection{Using the counter to unlock the side path}

% - change condition on ``more about the hood'' to ``number fact counter = input
% 2''
We can now make use of this counter to unlock the ``more about the 
hood'' side path after the reader has gone through the story twice, 
rather than just once.

\begin{enumerate}
    \item Edit the ``more about the hood'' link in the ``start'' 
    node, and edit the ``more about the hood'' rule. This rule was 
    added in Tutorial 1. It contains one condition, ``Node end 
    visited'', and one action, ``follow link to Hood Details''. 
    \item We 
    want to retain the existing action, but instead of triggering the 
    rule if the ``end'' node has been visited, we want to trigger it 
    only if the ``end'' node has been visited two times.
    To do this, we will use our ``counter'' fact, and use a number 
    fact condition to check the value of the fact. 
    
    Change the type of 
    the condition to ``Number Fact'', and choose fact ``counter''.
    \item We want to check whether the reader has visited the ``end'' 
    node twice, so set the comparator pulldown menu to show ``='', 
    and make sure the type pulldown menu for the right-hand side is 
    set to ``Input''. Then type the value ``2'' in the text entry field.
\end{enumerate}

% - run - what if read 3 times?
% - instead try $\ge$ input 2
\noindent Now try running the story. Read through once, and then go back to the 
``start'' node by clicking on the ``back to start'' link in the 
``end'' node. Notice that the link on the text ``red hood'' is not 
clickable. Now continue to read, and once again to back to the start. 
This time, the second time you've gone back to the start, the ``red 
hood'' link \textit{is} clickable.

What do you think will happen if you read through again, and then go 
back to the start a \textit{third} time? Try it. Why isn't the link 
clickable the third time?

Remember, we set the condition on the link to be ``Number fact 
counter = 2''. This means that the link is clickable when ``counter'' 
is \textit{exactly} equal to 2. This is fine if we want the link to 
only be available on the second rereading. If, however, we want it to 
be unlocked for every subsequent rereading, we should use ``$\ge$'' 
instead of ``=''. Try this, and then read the story again. This time, 
you should see the ``red hood'' link is clickable for the second and 
subsequent rereadings.

% leave this as an exercise in ``next steps'' below :)
% make it so that it doesn't let you re-read more than 3 times
% - show problem with putting 2 separate rules on ``restart'' link (LRRH5d.dyn
% and LRRH5e.dyn)
%There is one last thing that we'll do. Since we have a counter, it is 
%possible to \textit{limit} the number of times that the reader can 
%reread the story. We will now limit the number of rereadings to 3, by 
%settting the ``back to start'' link in the ``end'' node to only be 
%clickable if counter $<$ 3, ie. if we have already reread (followed the 
%``back to start'' node) less than three times.

% maybe (later?) do something about procedural conversations? to show
% complexity?

\section{Next steps}

We have modified the original story from Tutorial 1, 
\textsc{LRRH.dyn}, by adding random selection of endings, and by 
adding a counter which both unlocks the side path on the second 
rereading. The completed version of this story can be found in the 
file \textsc{LRRH5.dyn}. 

There are several things that you could try to enhance the story. For example,
you limit the number of rereadings to three, making sure that once 
you prevent the reader from rereading you let them know why. You 
could also make sure that the reader doesn't get the same random 
ending twice in a row. One way of adding these enhancements can be 
found in \textsc{LRRH6.dyn}. 

\section{Conclusion}

In this tutorial, we have created a simple hypertext fiction which 
makes use of randomization and a counter to procedurally alter the 
story which the reader encounters. This provides the basis for much 
more procedural approaches to creating interactive stories. See what 
you can think of, and then try it!

\end{document}