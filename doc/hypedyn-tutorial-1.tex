\documentclass{article}
\usepackage{graphicx}
\usepackage{hyperref}

\begin{document}

\title{The HypeDyn Hypertext Fiction Editor\\Tutorial 1: Nodes, Links and Rules}
% \author{Alex Mitchell}
\date{}

\onecolumn
\maketitle

\tableofcontents


\section{Introduction}
In this tutorial, we will build a basic interactive story in the HypeDyn
hypertext fiction editor. HypeDyn is a tool which lets you create ``hypertext'',
a form of writing which usually consists of:

\begin{itemize}
  \item A collection of text fragments or ``nodes'', containing some text, which
  are normally displayed one at a time in a hypertext reader;
  \item ``links'' between these nodes, which provide a connection from one node,
  or more commonly from a section of text in one node, to another node. A reader
  is usually able to ``follow'' a link to its destination node by, for example,
  clicking on the link; 
  \item ``rules'' which can be attached to a link, which determine whether or
  not the link can be followed, usually based on some conditions.
\end{itemize}

In this tutorial, we will be creating a simple hypertext \textit{fiction}.
Hypertext fiction usually consists of a story told in hypertext. Most well-known
examples of hypertext fiction let the reader explore through a set of text
fragments, encouraging them to discover different paths through the story. These
alternate paths sometimes uncover new information which the reader didn't
previously know, thereby altering their perception of the story.

Our example, ``Little Red Riding Hood'', involves the reader exploring several
paths through a story. The main path tells a simplified version of the
traditional folk tale, about a little girl with a red hood who meets a wolf on
the way to visit her Grandmother. An addtional side path, which can only be accessed once
the reader has visited the last node in the main path, reveals some additional
background information about Red's hood.


\begin{figure}[ht]
  \centering
  \includegraphics[width=10cm]{images/hypedyn-tutorial-1-figure-1}
  \caption{\textit{The completed ``Little Red Riding Hood'' story.}}
\end{figure} 

The nodes and links in the final story are shown in Figure 1. Although this is
a very simple story, it will involve most of the basic functionality of HypeDyn.

\textit{Note:  HypeDyn is a work-in-progress, so there are some features that are still
not completed, and there may be bugs. If you encounter any errors, please
report them as bugs on our Launchpad site: \url{https://launchpad.net/hypedyn}.}

\section{Getting started}

First, open HypeDyn by double-clicking on the file \textbf{HypeDyn.exe} (in
Windows) or \textbf{HypeDyn.app} (in MacOS).

Save your file by choosing Save from the File menu. Make sure that you give the
file a \textbf{.dyn} extension.

\section{Creating a text node}

The first thing we need to do is to start creating some text nodes. We'll begin
by creating the starting node, which represents the first fragment in the story.
In this node, we will introduce the main character, Red, and the initial action of
the story, the fact that she is walking through the forest.
 
\begin{figure}[ht]
  \centering
  \includegraphics[width=10cm]{images/hypedyn-tutorial-1-figure-2}
  \caption{\textit{The main HypeDyn interface}}
\end{figure} 

The main HypeDyn interface consists of the Toolbar (A), the Node list (B), and
the Map view (C), as shown in Figure 2. The Toolbar contains the functions that
we will use to create and edit our text nodes. The Node list contains a list of
the nodes in your story. The Map view displays the nodes and links.

\begin{enumerate}
  \item To create a new text node, click on the New node button. A new text node
  will be created, and you will be asked to give it a title. Name the new text
  node ``Start'', and click Ok, as shown in Figure 3.  
\end{enumerate}

 
\begin{figure}[ht]
  \centering
  \includegraphics[width=4cm]{images/hypedyn-tutorial-1-figure-3}
  \caption{\textit{Creating a new text node}}
\end{figure} 

Text nodes are the basic building blocks of hypertexts created with HypeDyn.
Each text node has: 
\begin{enumerate}
  \item A title to identify the node
  \item Some text content.
\end{enumerate}

Once you've created a text node, you should see the node represented in both
the node list and the Map view (see Figure 4). You can select the node by
clicking on its title in the node list. 
 
\begin{figure}[ht]
  \centering
  \includegraphics[width=7cm]{images/hypedyn-tutorial-1-figure-4}
  \caption{\textit{Our first node showing in the Node list and Map view}}
\end{figure} 

You can also drag the node around in the Map view by selecting node dragging
it. Once we have created several nodes and links between them, it can be useful
to position them in the Map view so that we can easily see the relationship
between them. However, the position of the nodes in the Map view does
not change their behaviour in any way.

\section{Adding text to a node}

Now that we have our first text node, we need to go in and enter some text.

\begin{enumerate}
  \item Choose the title of our node, start, in the node list, and then click
  on the Edit node button in the Toolbar. This will open the editing view for
  the node, as shown in Figure 5. 
  
\begin{figure}[ht]
  \centering
  \includegraphics[width=9cm]{images/hypedyn-tutorial-1-figure-5}
  \caption{\textit{The editing view of a text node}}
\end{figure} 

The Editing view lets you enter the text for a node. It also lets you add, edit
and delete links, and test your story by using the read button to read from the
current node. The Editing view consists of the Toolbar (A), the Link list (B),
and the Content editor (C).

Note that the name of the node that you are currently editing is shown in the
title bar of the Editing view.

\item Now enter the text for the first fragment of our story in the Content
editor:

\begin{quotation}
This is the story of a girl named Little Red Riding Hood, which she was called
because of the red hood that she often wore.

One day she was walking through the forest.
\end{quotation}

\item We want this node to be the start of our story, so click on the Set start
node button to set this node as the start node. Every story must have a start node.
\item Once you've finished entering this text, close the editing view by
clicking on the close button. This will take you back to the Map view.
\end{enumerate}

\section{Creating links between nodes}

We will now create several other nodes, to represent other fragments of our
hypertext story, and we'll create links between them.

\subsection{Creating the other nodes}

\begin{enumerate}
  \item Create another text node, and name it ``forest''.
  \item Enter the following text in the node:

\begin{quotation}
In the forest, Red came across a young man with a nasty smile.

"Where are you going, little girl?" he asked.

"I'm off to see my sick granny," she said.

Well, you can probably guess what happened next.
\end{quotation}

This node represents the main conflict of the story.

\item Create a third node, and name it ``end''.
\item Enter the following text in the node:

\begin{quotation}
*** The End ***

back to start
\end{quotation}

This node represents the end of our story.
\end{enumerate}

Now you should have three nodes: ``start'', ``forest'', and ``end'', as in Figure 6.

 
\begin{figure}[ht]
  \centering
  \includegraphics[width=7cm]{images/hypedyn-tutorial-1-figure-6}
  \caption{\textit{Three text nodes}}
\end{figure} 

\subsection{Creating a link}

Now we want the user to be able to move from one text node to another, as they
make decisions about how to read the story. To do this, we need to create links
between the nodes.

Links in HypeDyn can be attached to a specific piece of text. In our story, we
want the user to be able to click on ``forest'' in the ``start'' node to jump
to the ``forest'' node. This lets the user progress through the main story path.

\begin{enumerate}
  \item Select the ``start'' node in the Node list, and then click on the Edit
  node button.
  \item Now select the word ``forest''. We'll attach our link here.
  \item Click on the New link button in the Editing view. A New link dialogue
  box will appear, as shown in Figure 7. 

 
\begin{figure}[ht]
  \centering
  \includegraphics[width=4cm]{images/hypedyn-tutorial-1-figure-7}
  \caption{\textit{Creating a link}}
\end{figure} 

\item Give the link a name, such as ``Next'', and click OK.
\item You will now see the ``Edit link'' dialogue (see Figure 8). This is where
we can specify the destination of the link. The ``Edit link'' dialogue consists
% NM2217 version:
% of two sections: the \textit{rule}, listing a series of conditions which
% specify when the link can be followed, and a \textit{THEN} section, which is the set of
% actions that are performed when the rule is satisfied. We will return to the
% \textit{rule} below.
% full version:
of three sections: the \textit{rule}, listing a series of conditions which
specify when the link can be followed, a \textit{THEN} section, which is the set of
actions that are performed when the rule is satisfied, and an \textit{ELSE} section,
which is a set of actions that are performed when the rule is \textit{not}
satisfied. In this tutorial, we will not be dealing with the \textit{ELSE}
section. For details on the \textit{ELSE} section, see tutorial 2. We will
return to the \textit{rule} below.

\begin{figure}[ht]
  \centering
  \includegraphics[width=6cm]{images/hypedyn-tutorial-1-figure-8}
  \caption{\textit{The link from the ``start'' to the ``forest''}}
\end{figure} 

\item Check the checkbox to the left of ``follow link to'' in the
\textit{THEN} section, and use the pull-down menu to choose the ``forest'' node.
This will create a link from the text ``forest'' in the ``start'' node to the
``forest'' text node.
\item Now click OK. 
\end{enumerate}

We have now created a link from the ``start'' node to the ``forest''. You
should be able to see this link in the Link list at the top of the Editing
view, and in the Map view, as shown in Figure 9.
  
\begin{figure}[ht]
  \centering
  \includegraphics[width=9cm]{images/hypedyn-tutorial-1-figure-9}
  \caption{\textit{The link from the ``start'' to the ``forest''}}
\end{figure} 

Notice that links show up as bold, underlined text in the content of the text node.

\subsection{Testing the link}

Now that we've created the link, we need to test that it works.

To switch from editing to reading the text nodes, we need to use the Run button
in the Toolbar in the Map view.

\begin{enumerate}
  \item Close the ``start'' node if its still open.
  \item Now click on the Run button. The Reader window will open, showing
  the contents of the text node. You are now ``reading'' the story, instead of
  editing it, as shown in Figure 10.

 
\begin{figure}[ht]
  \centering
  \includegraphics[width=9cm]{images/hypedyn-tutorial-1-figure-10}
  \caption{\textit{Reading a text node}}
\end{figure} 

In the Reader, you can click on links, and you'll be taken to the text node
that the link leads to.

\item Click on the word ``forest'' to follow the link.
\item The Reader will change to show the contents of the ``forest'' node.
\end{enumerate}

You can finish reading the story by either closing the Reader window, or
clicking on the Stop button in the Toolbar of the main window.

We now have links between nodes, the main functionality of a hypertext document.

% this should be fixed now
% \textit{Note: For now, be careful not to give two links the same names, even if they
% are in different nodes. There used to be a bug in HypeDyn which makes it
% difficult to delete links with the same name. I think this has been fixed, but
% its better to be safe than sorry\ldots}

\section{Creating more links}

Now that we have one link from the ``start'' to the ``forest'', we should link
in our third node, ``end''. This will allow the user to move forward along the
main story path from the ``forest'' node to the ``end'' node, the end of the story.

\begin{enumerate}
\item Open the ``forest'' text node.
\item Select the word ``next''.
\item Click on the New link button.
\item Name this link ``to the end''.
\item Now select the ``end'' node in the pull-down menu.
\item Click on the Ok button.

You now have a link to the ``end'' text node. Test this out by opening the
Reader with the ``forest'' node selected. You should be able to move to the
``end'' node. 

 
\begin{figure}[ht]
  \centering
  \includegraphics[width=6cm]{images/hypedyn-tutorial-1-figure-11}
  \caption{\textit{Three nodes with a loop back to the start}}
\end{figure} 

Now we'll create one more link, this time from the ``end'' node back to the
``start'', so that the reader can re-read our story if they want.

\item Open the ``end'' node.
\item Select the words ``back to start''.
\item Click on the New link button.
\item Name this link ``restart''.
\item Now select the ``start'' node from the pull-down menu.
\end{enumerate}

We now have a loop back to the start. At this point, your Map view should look
as shown in Figure 11.

\section{Creating rules}

The next thing we want to do is create a rule. This is a link that can only be
followed if a certain condition, such as the user having visited a specific
node, is satisfied.

We will create a node, named ``Hood details'', which can be reached after the
reader has read the entire story. This node will reveal additional information
that will (hopefully) change the reader's experience of the story, as the reader
will now know that there is something unusual, and perhaps slightly sinister,
about Red's hood.

To do this, we will use a feature in HypeDyn called a ``rule'' to only allow the
reader to see the ``Hood details'' node after they have seen the ``end'' node.
This allows us, as the author of the story, to ensure that the reader has gone
through the main story path at least once. We want the reader to experience the
original story before seeing the new information, so that the new information
can more effectively impact their previous mental model of the events of the
story.

\subsection{Creating the node}

First we'll create a new node to contain the information about the hood.

\begin{enumerate}
  \item Create a new node, and name it ``Hood details''.
  \item Enter the following text in the text node:

\begin{quotation}
Her hood was a magic garment, given to her by her grandmother.
It could kill anyone who tried to harm the wearer.

But not immediately. And in a most painful manner.

Meanwhile, in the forest...
\end{quotation}

\item Next, select the word ``forest'' in the text you just entered (in the 
``Hood details'' node). Create a link from this word to the ``forest'' node.
Name this link ``continue''. 

This link leads the reader back to the main story path, thus allowing the
reader to continue reading the story after having read the additional details in
this node.
\item Now create a link from the ``start'' node to the ``Hood details'' node.
Edit the ``start'' node and select the text ``red hood''. Create a link from
this text to the ``Hood details'' node, and name the link ``more about the hood''.
\end{enumerate}

At this point, you should have the nodes as shown in Figure 12.

 
\begin{figure}[ht]
  \centering
  \includegraphics[width=7cm]{images/hypedyn-tutorial-1-figure-12}
  \caption{\textit{Setting up the ``Hood details'' node}}
\end{figure} 

\subsection{Creating the rule}

Now we want to make it so that the reader can only follow the ``more about the
hood'' link if they've already read the ``end'' node. To do this, we need to
create a rule on the link between the ``start'' node and the ``Hood details'' node.

\begin{enumerate}
  \item Open the ``start'' text node, and select the link ``More about the hood''.
  \item Now click on the Edit link button. The Edit link dialog box will
  appear.

The Edit link dialog box lets us create Conditions which must be true for the
link to be followed. If there are multiple conditions, then you can choose
whether the link is followed when All or when Any of the conditions are true,
by using the Any/All pull-down menu. For now, leave this pull-down menu at the
default, All.

\item Click on the Add condition button. This will create a new condition which
must be satisfied for the link to be followed, as shown in Figure 13.
 
\begin{figure}[ht]
  \centering
  \includegraphics[width=7cm]{images/hypedyn-tutorial-1-figure-13}
  \caption{\textit{Adding a condition}}
\end{figure} 

\item We want the link to ``Hood details'' only to be followed if the reader has
gone through the entire story at least once. This means they have to have
visited the ``end'' node. Use the pull-down menus so that our condtion reads
``Node end Visited''. (Your condition should look like Figure 14.) Then
click Ok.

\begin{figure}[ht]
  \centering
  \includegraphics[width=7cm]{images/hypedyn-tutorial-1-figure-14}
  \caption{\textit{Condition for the ``Hood details''}}
\end{figure} 

\item Close the ``start'' text node.
\end{enumerate}

Note that the check-box beside the condition is used to select the condition if you
want to delete it. Clicking on ``Delete selected'' will delete any selected
conditions. Only select a condition if you want to delete it.


\subsection{Testing the rule}

Now we will test out the rule.

\begin{enumerate}
  \item In the main window, click on the Run button.
  \item In the Reader window, you should see that the text ``red hood'' is bold,
  but not underlined. This indicates a link which cannot currently be
  followed, as shown in Figure 15.

 
\begin{figure}[ht]
  \centering
  \includegraphics[width=7cm]{images/hypedyn-tutorial-1-figure-15}
  \caption{\textit{Start node with a link that cannot be followed}}
\end{figure} 

\item Now read through the story once, by clicking on ``forest'' in the
``start'' node, and then ``next'' in the ``forest'' node.
\item Finally, read back to the start by clicking on the ``back to start'' link
in the ``end'' node.
\item This time, you should see that the ``red hood'' text is a normal,
followable link. Click on the link to go to the ``Hood details'' node.

If you want to read the story again as if you've never read it, you need to
clear the Reader's history of which nodes have been visited.

\item In the Reader, click on the Restart. The Reader will go to the node which
we set as the start node, ``start'', and will clear its history.
\item You should now see that the ``red hood'' text is once again an
unfollowable link.
\end{enumerate}

We have now created a simple story, with links and rules.

\section{Reading the story}

Once you have created and saved your story, you can use the stand-alone reader,
\textbf{HypeDynReader.exe} (on Windows) or \textbf{HypeDynReader.app} (on
MacOS) to read the story, by running the stand-alone reader and then opening the
story. This is exactly the same as the Reader window in the editor.

\section{Conclusion}

In this tutorial, we have created a simple hypertext fiction. Our story has a
main path, which tells a simplified version of the traditional folk tale
``Little Red Riding Hood''. Our story also has an alternative side path, which
gives the reader more information about Red's hood. This alternative side path
can only be read once the reader has visited the ``end'' node at the end of the
main story path. Although this is a simple story, it demonstrates all the
capabilities of HypeDyn, capabilities which are sufficient to create a more
complex hypertext fiction.

% \section{Final notes}

% As mentioned above, the HypeDyn hypertext editor is still a work-in-progress.
% Please let me know if you encounter any problems. Also, please post your
% questions about using the tool to the IVLE forum.

Please see tutorial 2 for more advanced features: alternate links, conditional
text, and anywhere nodes.

\end{document}
