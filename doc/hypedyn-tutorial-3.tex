\documentclass{article}
\usepackage{graphicx}
\usepackage{hyperref}

\begin{document}

\title{The HypeDyn Hypertext Fiction Editor\\Tutorial 3: Node Rules and Facts}
% \author{Alex Mitchell}
\date{}

\onecolumn
\maketitle

\tableofcontents

\section{Introduction}
In this tutorial, we will introduce some new features to HypeDyn:

\begin{itemize}
  \item ``node rules'', which allow conditions to be attached to nodes as well
  as links, and allow you to update facts when a node is visited; and
  \item ``facts'', which represent information which can be set and checked
  while a story is being read. Facts can be used in conditions, and can also be
  used in conditional text.
\end{itemize}

% We will also be working with a modified version of HypeDyn, which only allows
% \textit{one} non-anywhere node. This is to allow you to focus on the idea of
% hypertext where you do \textit{not} specify links explicitly - instead, you set
% conditions as to when anywhere nodes can be seen using node rules and facts.

% - t/f facts are actually a convenience, could do the same with node/link
% conditions - except, can retract facts selectively, and can simplify if several
% nodes have same outcome
% - text facts also are simply a convenience to reduce using may diff alt texts
% - attempt to encourage procedural thinking
% - node rules let you set facts
% - also let you use anywhere nodes conditionally - sculptural approach
% [add count? add time? add numbers/math?]

\noindent We will be creating a new version of the ``Little Red Riding Hood''
story, which does not build on the versions created in tutorials 1 and 2. This
time, we will be using \textit{anywhere} nodes for all of the content, and will
be making use of the automatic generation of links to these nodes, plus a
combination of \textit{facts} and \textit{node rules} to control when the reader
can see these nodes. The nodes and facts in the final story are shown in Figure
\ref{fig:tut3:completed}.

\begin{figure}[h]
  \centering
  \includegraphics[width=10cm]{images/hypedyn-tutorial-3-figure-1}
  \caption{\textit{The completed ``Little Red Riding Hood'' story.}}
  \label{fig:tut3:completed}
\end{figure} 

\textit{Note:  HypeDyn is a work-in-progress, so there are some features that are still
not completed, and there may be bugs. If you encounter any errors, please
report them as bugs on our Launchpad site: \url{https://launchpad.net/hypedyn}.}

\section{Getting started}

% Make sure that you are working with HypeDyn 2.1s, the customized version of
% HypeDyn for NM3222 project 2.

First, open HypeDyn by double-clicking on the file \textbf{HypeDyn.exe} (in
Windows) or \textbf{HypeDyn.app} (in MacOS). Save your file by choosing ``Save''
from the File menu. Make sure that you give the file a \textbf{.dyn} extension.

% Notice that there are some subtle differences from the version of HypeDyn you
% used in tutorials 1 and 2 (see Figure 2). The node list (A) and graph view (D and
% E) are the same. However, notice that initially the normal node graph view is
% much smaller than the anywhere node graph view. This reflects the fact that you
% will be writing your story entirely with anywhere nodes.
% 
% Next, notice that there is a new list view below the node list (B) - this is
% the \textit{fact list}. There is also a new menu (C) - this is the \textit{fact
% menu}. We will come back to these later in the tutorial.
% 
% \begin{figure}[h]
%   \centering
%   \includegraphics[width=10cm]{images/hypedyn-tutorial-3-figure-2}
%   \caption{\textit{The main HypeDyn window.}}
% \end{figure} 
% 
% Notice that when you open HypeDyn, the new file automatically has one node -
% a normal node, which is automatically named ``Start'' and set to be the start
% node. Also notice that the \textit{New Node} button is gone. The \textit{New Node} menu
% item in the \textit{Nodes} menu is also gone. If you select the ``Start''
% node, you'll see that the \textit{Delete node} button is disabled, and the
% \textit{Delete node} menu item is also disabled. In this version of HypeDyn, you
% cannot create normal nodes, and you cannot delete the one node that was created
% for you. All of your story content must be written in \textit{Anywhere} nodes.

\section{Writing ``sculptural'' hypertext}

For this tutorial, we are going to create a story which consists almost entirely
of \textit{anywhere nodes}. The only normal node will be the ``start'' node.
Writing a story this way requires a slightly different way of thinking about
hypertext. Each node, as in regular hypertext, represents a fragment of the
story. Since the nodes you'll be using are anywhere nodes, initially all the
nodes will be accessible from every other node - essentially there are implicit
links between every node.

Once your nodes are created, you need to start thinking about restricting the
reader's possible paths through the nodes, otherwise your story will have no
structure, and the reader is likely to become overwhelmed and lost. You will do
this by using \textit{node rules} to determine when an anywhere node's link can
be seen. As you start restricting access, you are essentially removing some of
the implicit links.

This approach is often referred to as ``sculptural'' hypertext,
\cite{Bernstein:2001aa,Bernstein:2002aa}, since the process of placing conditions
on the nodes is similar to carving out a sculpture from a block of stone - what
is left behind once unwanted links are carved away is the story. We'll see how
this works as we go through and create our version of Little Red Riding Hood.

\section{Node rules}

We are now going to create the outline of our story, and then create node rules
to control the order in which readers can access the nodes.

\subsection{Creating the nodes}

We'll start by creating a collection of nodes which represent the fragments in
the story. After creating these nodes, you will have something like what can be
seen in Figure \ref{fig:tut3:initial_nodes}.

\begin{figure}[h]
  \centering \includegraphics[width=10cm]{images/hypedyn-tutorial-3-figure-2}
  \caption{\textit{The initial set of nodes.}}
  \label{fig:tut3:initial_nodes}
\end{figure} 

Note that, as there are no links, the arrangement of the nodes in the graph view
is entirely up to you. Also note that you can drag the border between the regular
and anywhere node portions of the map view so that you can see all of the
anywhere nodes.

\begin{enumerate}
  \item First, create a node named ``start''.
  \item Edit the node, and make it the start node by clicking on the \textbf{Set
  start node} button.
  \item Now add the following text:

\begin{quotation}
\noindent One day, Little Red Riding Hood is walking through the forest, on the
way to deliver a basket of food and flowers to her grandmother. 
\end{quotation}

\item Now we need to create the rest of the story content, which will consist
of a set of anywhere nodes. Create 7 anywhere nodes, with names and content as
shown in Table \ref{table:tut3:node_content}.

\end{enumerate}

\begin{table}[h]
  \centering
  \begin{tabular}{| p{3cm} | p{8cm} |}
    \hline
    Name & Content \\
    \hline
    \hline
    Explore the forest & 
    Tempted by a grove of flowers, Red strays off the path into the forest.
    There are a number of different types of flowers growing in the grassy
    clearing. \\
    \hline
    Go deeper into the forest &
    A handsome young man is leaning against the trunk of a tree. He gestures to
    Red to come over. \\
    \hline
    Talk to young man &
    Red goes over and talks to the wolf. He asks her where she's going, and she
    says she's off to deliver a basket of food and flowers to her sick grandma.
    \\
    \hline
    Go directly to Grandma's house &
    Red walks along the path, sticking carefully to the center to avoid the
    dark, menacing trees. Eventually, she reaches Grandma's house. 
  
    \bigskip
  
    When Red enters Grandma's house, she is surprised to see the young man
    sitting on the sofa.
  
    \bigskip
  
    Grandma smiles when she sees Red. \\
    \hline
    Pass the basket to Grandma &
    Red passes the basket of food and flowers to Grandma. \\
    \hline
    Approach the young man &
    Unfortunately, the young man was a wolf. Neither Red nor Grandma were ever
    seen again. \\
    \hline
    Head home &
    Red heads back home. \\
    \hline
  \end{tabular}
  \caption{Node contents}
  \label{table:tut3:node_content}
\end{table}

If you now try running the story, you'll see that all the nodes are available
from all other nodes (see Figure \ref{fig:tut3:running_start}). If you click on
the links, at each node all the other nodes are available. We will now start to
carve out the shape of our story using \textit{node rules}.

\begin{figure}[h]
  \centering
  \includegraphics[width=8cm]{images/hypedyn-tutorial-3-figure-3}
  \caption{\textit{Links from every node to every other node.}}
  \label{fig:tut3:running_start}
\end{figure} 

\subsection{Editing a node rule}

We will now explore the use of \textit{node rules}. Edit the node ``Talk to young
man'', and click on the ``Edit node rule'' button. You will see the \textit{Edit
node rules} window (see Figure \ref{fig:tut3:edit_node_rules}).

\begin{figure}[h]
  \centering
  \includegraphics[width=8cm]{images/hypedyn-tutorial-3-figure-4}
  \caption{\textit{The edit node rules window.}}
  \label{fig:tut3:edit_node_rules}
\end{figure} 

Notice that this window looks similar to the \textit{Edit Link} window that you
would have seen in tutorials 1 and 2. Both windows contain a list of rules, which
are evaluated in order. The main difference is that for node rules, the rules are
generally evaluated when the reader \textit{enters} the node. For link rules,
rules are generally evaluated when the reader \textit{clicks} on the
link\footnote{Evaluation of link rules is actually done in two steps: ``update
text using'' actions are carried out when the reader enters the node containing
the link, and all other actions are carried out when the reader clicks on the
link.}.

For an anywhere node, a default rule is created when the node is first created,
named ``Add Anywhere Link''. Edit this rule by selecting it and then clicking
on the ``Edit Rule'' button. You will see the \textit{Edit rule} window (see
Figure \ref{fig:tut3:edit_rule}).

\begin{figure}[h]
  \centering
  \includegraphics[width=8cm]{images/hypedyn-tutorial-3-figure-5}
  \caption{\textit{The edit rule window for a node rule.}}
  \label{fig:tut3:edit_rule}
\end{figure} 

Again, this window looks similar to the corresponding ``Edit Rule'' window for
link rules. In fact, both windows are rule editors. Rules in HypeDyn consist of
two parts: a set of \textit{conditions}, and a set of actions which are carried
out if the conditions are true. In a node rule, the conditions are the same as
for a link. The actions, however, are different. Notice that the action list
already contains a default action: ``enable links to this node from anywhere''.
The other action which is available is ``update fact''. We will look at the first
of these actions now, and come back to the second later in the tutorial.

\subsection{Controlling access with node rules}

What we want to do at this point is control access to our anywhere nodes. We
can do this by setting conditions in the node rule. If these conditions are
true, then HypeDyn will enable the automatic anywhere links to this node.
Otherwise, the node will not be accessible.

So, what we need to do is think carefully about \textit{when} each of our nodes
should be accessible. One way to think about this is to consider the conditions
on the node to be \textit{preconditions} which must be satisfied for this event
(node) to take place. These preconditions can be expressed in terms of which
other nodes the reader has already seen. For example, for the ``Talk to young
man'' node, we might want this only to be available to the reader if Red has
gone deeper into the forest, ie. the reader has seen the ``Go deeper into the
forest'' node. Lets add this as a condition.

\begin{enumerate}
  \item Click on ``Add condition''.
  \item As in a link, a new condition will appear. If you pull down the
  \textit{Node} pulldown menu, you'll see three options: \textit{Node},
  \textit{Link}, and \textit{Fact}. The first two you've seen before. We'll
  return to the third later.
  \item Choose \textit{Node}, and then choose \textit{Go deeper into the
  forest} and \textit{Visited}.
\end{enumerate}

If you try running the story now, you'll see that although all the other nodes
are still accessible, the node ``Talk to young man'' isn't available until
you've visited ``Go deeper into the forest''.

There is a problem, though - after Red has talked to the young man, and the
reader clicks on any of the other links, the ``Talk to young man'' link
reappears. This can lead to the reader repeatedly looping through the same
node. Although this may sometimes be desirable, particularly if you are using
alternative text to procedurally change the contents of the node, in our story
we don't want this. To prevent this, we can add one more condition to our node:
that the ``Talk to young man'' node itself has \textit{not} been visited.

\begin{enumerate}
  \item Edit the node rule for ``Talk to young man''.
  \item In the ``Add Anywhere Link'' rule, add another condition, as follows:
  ``Node Talk to young man Not Visited''.
\end{enumerate}

\begin{figure}[h]
  \centering
  \includegraphics[width=10cm]{images/hypedyn-tutorial-3-figure-6}
  \caption{\textit{The node rule for ``Talk to young man''.}}
  \label{fig:tut3:talk}
\end{figure} 

Your node rule should be similar to what is shown in Figure \ref{fig:tut3:talk}.

Now if you run the story, the ``Talk to young man'' node is \textit{only}
available after you visit ``Go deeper into the forest'' and haven't yet
visited ``Talk to young man''.

This handles most of the conditions. However, what if the reader had decided
to, for example, click on ``Head home'' or ``Go directly to Grandma's house''?
Would it make sense for Red to still be able to talk to the young man at this
point? Probably not, so we need to add two more conditions.

\begin{enumerate}
  \item Edit the ``Add Anywhere Link'' rule for ``Talk to young man''.
  \item Add another condition, as follows: ``Node Head home Not
  Visited''.
  \item Finally, add the condition ``Go directly to Grandma's house Not
  Visited''.
\end{enumerate}

\begin{figure}[h]
  \centering
  \includegraphics[width=10cm]{images/hypedyn-tutorial-3-figure-7}
  \caption{\textit{The final node rule for ``Talk to young man''.}}
  \label{fig:tut3:talk_final}
\end{figure} 

Your final node rule should be similar to what is shown in Figure
\ref{fig:tut3:talk_final}.

\subsection{Adding the rest of the node rules}

We can now go through and consider each of the other nodes, and add appropriate
conditions. A good way to do this is to work back from the most constrained
nodes to the least constrained nodes. So, working back from ``Talk to young
man'', we can now consider ``Go deeper into the forest''. Logically, this node
should only be available if the reader has chosen to visit ``Explore the
forest'', and should only be seen once. It should also not be available if the
reader has chosen ``Head home'' or ``Go directly to Grandma's house''. To get
this to work, we can follow a similar approach to what we did above.

\begin{enumerate}
  \item Edit the ``Add Anywhere Link'' node rule for ``Go deeper into the
  forest.
  \item Add the following conditions:
  \begin{enumerate}
  \item``Node Explore the forest Visited''.
  \item ``Node Go deeper into the forest Not Visited''.
  \item ``Node Head home Not Visited''.
  \item ``Go directly to Grandma's house Not Visited''.
  \end{enumerate}
\end{enumerate}

The next node we'll consider is ``Explore the forest''. This node is much less
restricted, as we want this to be available to the reader immediately, but only
if it has not yet been visited. As with ``Go deeper into the forest, it should
also not be available if the reader has chosen ``Head home'' or ``Go directly
to Grandma's house''.

\begin{enumerate}
  \item Edit the ``Add Anywhere Link'' node rule for ``Explore the forest''.
  \item Add the following conditions: 
  \begin{enumerate}
  \item ``Node Explore the forest Not Visited''.
  \item ``Node Head home Not Visited''.
  \item ``Go directly to Grandma's house Not Visited''.
\end{enumerate}
\end{enumerate}

We have now set out a path for the reader from the start through to talking to
the young man. Next, lets turn our attention to the path to Grandma's house.
The node ``Go directly to Grandma's house'' should always be available
\textit{unless} the reader has already seen it, or has decided to view ``Head
home''.

\begin{enumerate}
  \item Edit the ``Add Anywhere Link'' node rule for ``Go directly to Grandma's
  house''.
  \item Add the following conditions: 
  \begin{enumerate}
  \item ``Node Head home Not Visited''.
  \item ``Go directly to Grandma's house Not Visited''.
\end{enumerate}
\end{enumerate}

Notice that there is also some text in this node which probably should have
conditions attached to it. Following what we did in the previous tutorials, set
it up so that the second paragraph only appears if Red talked to the young man,
and the third paragraph only appears if she \textit{didn't} talk to him.

There are three nodes left: ``Pass the basket to Grandma'', ``Approach the
young man'', and ``Head home''. The first two are important, since they depend
on whether or not Red told the young man where she is going. Because of this,
we need to carefully design their conditions.

For ``Pass the basket to Grandma'', we want the node to be available only if
Red has gone to Grandma's, she \textit{didn't} talk to the wolf, she hasn't gone
home, and she hasn't yet passed the basket to Grandma. This translates into the
following conditions:

\begin{enumerate}
  \item Edit the ``Add Anywhere Link'' node rule for ``Pass the basket to
  Grandma''.
  \item Add the following conditions:
  \begin{enumerate}
  \item ``Go directly to Grandma's house Visited''.
  \item ``Node Talk to young man Not Visited''.
  \item ``Node Head home Not Visited''.
  \item ``Node Pass the basket to Grandma Not Visited''.
\end{enumerate}
\end{enumerate}

For ``Approach the young man'' we have a similar set of conditions, although
this time we want to specify that Red \textit{did} talk to the young man.

\begin{enumerate}
  \item Edit the ``Add Anywhere Link'' node rule for ``Approach the young man''.
  \item Add the following conditions:
  \begin{enumerate}
  \item ``Go directly to Grandma's house
  Visited''.
  \item ``Node Talk to young man Visited''.
  \item ``Node Head home Not Visited''.
  \item ``Node Approach the young man Not Visited''.
\end{enumerate}
\end{enumerate}

Finally, we have to set the conditions for ``Head home''. This is the least
constrained of all the nodes, as it should be available everywhere
\textit{except} if Red and Grandma were eaten by the wolf.

\begin{enumerate}
  \item Edit the ``Add Anywhere Link'' node rule for ``Head home''.
  \item Add the condition: ``Node Approach the young man Not Visited''.
\end{enumerate}

Now run the story. You should be able to explore several variations of the
story, and see that the conditions that we specified ensure that our events
only occur if their preconditions are satisfied.

\subsection{Disabling the back button}

There is one remaining problem: the reader can click on the ``back'' button and
return to the previous node. In some cases, this does not make sense. For
example, after clicking on ``Talk to young man'', if the reader clicks on the
``back'' button, she will see the ``Go deeper into the forest'' node. We could
either have the contents of the nodes adapt to whether or not the reader has
seen the next node, which is quite complicated, or we could \textit{disable}
the back button. We'll do the latter.

To disable the back button, go to the File menu and choose Properties. You should
see the ``Properties'' dialog (see Figure \ref{fig:tut3:preferences}). Check the
checkbox to the left of ``Disable back button'', and click on the ``Ok'' button.
Now try running the story. You should see that the back button has been removed.

\begin{figure}[h]
  \centering
  \includegraphics[width=4cm]{images/hypedyn-tutorial-3-figure-8}
  \caption{\textit{Disabling the back button.}}
  \label{fig:tut3:preferences}
\end{figure} 

Disabling the back button is useful for cases such as this, where the procedural
nature of the story makes the back button problematic. It can also be useful if
you want to force readers to commit to their choices. Note, however, that it can
be frustrating for readers not to be able to go back, so use this feature with
care. Also note that preferences are saved with the story file, and only apply to
a specific story, so if you create a new story, the preferences will be set back
to their default settings.

\subsection{Sculptural vs. regular hypertext}

One thing to note is that we could have achieved a very similar effect using
regular nodes, links, and conditions. However, the \textit{process} of
developing a story using this form of hypertext is very different from what we
saw in tutorials 1 and 2. As a result, the type of story that you are likely to
create is also much different.

It would also be possible to reduce the complexity of the conditions that we
created above by making use of \textit{facts} to keep track of important
information in the story, and use this information to control when nodes are
available. We will now look at how to use \textit{facts}.

\section{Using \textit{facts} to keep track of what happened}

In previous tutorials, we have seen that HypeDyn can keep track of what the
reader has done by referring to which nodes have been visited, and which links
have been followed. This allows for quite complex procedural change. However,
one major limitaton is that you, as the author, are unable to have HypeDyn
\textit{forget} that a node was visited or a link was followed. There are also
times when a condition depends on one of several nodes having been visited,
which can lead to fairly complex conditions. You also might want to have a
condition that depends on, for example, the reader visiting a node several
times, which requires that you count the number of visits to a node.

To handle these limitations, we will now introduce \textit{facts}. In HypeDyn, a
fact is something which is important to the story. There are three types of
facts: \textit{true/false} facts, \textit{text} facts, and number facts.
\textit{True/false} facts are either \textit{true} or \textit{false}, and can be
checked in a condition and updated by an action. \textit{Text} facts contain a
piece of text, such as a sentence, can be updated by an action, and can be used
to replace text in a node in the same way as \textit{alternative text}.
\textit{Number} facts contain integers, such as 1 or 42, can be checked in a
condition, updated by an action, and can be used to replace text in a node. In
this tutorial we will cover true/false and text facts. See tutorial 4 for
details on how to use number facts.

\subsection{Creating the nodes}

Suppose we want to let Red pick the flowers in the forest grove, and put them
in the basket for Grandma. We could do this by using conditions and conditional
text. However, we will now show how this can be done with facts, and then
explain how this is actually more flexible than the approaches used in
tutorials 1 and 2.

First, create two new anywhere nodes:

\begin{enumerate}
  \item \textit{Name}: Pick the geraniums\\
  \textit{Content}: 
  \begin{quotation}
  \noindent Red decides to pick some of the geraniums in the grove and exchange
  them for the flowers in the basket for Grandma.
  \end{quotation}
  \item \textit{Name}: Pick the violets\\
  \textit{Content}: 
  \begin{quotation}
  \noindent Red decides to pick some of the violets in the grove and exchange
  them for the flowers in the basket for Grandma.
  \end{quotation}
\end{enumerate}

\subsection{Creating the facts}

We will now create two facts to keep track of whether or not Red has picked the
flowers and which flowers she picked, and one fact to keep track of whether or
not Red is in the forest grove (and therefore able to pick flowers).

\begin{enumerate}
  \item In the main HypeDyn window, go to the \textit{Fact} menu, and pick the
  \textit{New} menu item, and the \textit{True/False} submenu item. 
  \item The ``New True/False Fact'' dialogue will appear (see Figure
  \ref{fig:tut3:newfact}). Enter the name of the fact as ``Picked flowers'', and
  click ``ok''.

\begin{figure}[h]
  \centering
  \includegraphics[width=4cm]{images/hypedyn-tutorial-3-figure-9}
  \caption{\textit{Creating a new fact.}}
  \label{fig:tut3:newfact}
\end{figure}

  \item Now go to the \textit{Fact} menu, and pick the
  \textit{New} menu item and the \textit{True/False} submenu item again. 
  \item The ``New True/False Fact'' dialogue will appear. Enter
  the name of the fact as ``In the forest grove'', and click ``ok''.
  \item Now go to the \textit{Fact} menu, and pick the
  \textit{New} menu item and the \textit{Text} submenu item. 
  \item The ``New Text Fact'' dialogue will appear. Enter
  the name of the fact as ``The flowers'', and click ``ok''.
\end{enumerate}

We will use the fact ``Picked flowers'' to remember whether or not the reader has
had Red pick the flowers, and ``The flowers'' to remember the type of flowers. We
will use the fact ``In the forest grove'' to make sure that the flowers can only
be picked in the forest grove. This could be done by checking which nodes have
and haven't been visited, but in this case using a fact is much simpler, and
allows for greater flexibility.

\subsection{Using facts in node rule conditions}

Now we need to make sure that the two nodes for picking the flowers are only
available when Red is in the forest grove, and hasn't yet picked the flowers.
We'll do this by using the two \textit{true/false} facts created above.

\begin{enumerate}
  \item Edit the ``Add Anywhere Link'' rule in the node rule for node ``Pick the
  violets''.
  \item Add a condition, and select \textit{True/False Fact} as the type. Now
  choose the fact \textit{Picked flowers} and value \textit{False}. This means
  that the condition is true when the fact \textit{Picked flowers} is false,
  ie. Red hasn't picked the flowers. Note that facts are always \textit{false}
  until they have been updated, so when the story is started, both
  \textit{Picked flowers} and \textit{In the forest grove} are false.
  \item Add another condition, and set it to ``True/False Fact In the forest
  grove True''. This means that the condition is true when Red is in the
  forest grove.
\end{enumerate}

You should have conditions as in Figure \ref{fig:tut3:picktheviolets}.

\begin{figure}[h]
  \centering
  \includegraphics[width=10cm]{images/hypedyn-tutorial-3-figure-10}
  \caption{\textit{Conditions for ``Pick the Violets''.}}
  \label{fig:tut3:picktheviolets}
\end{figure}

\subsection{Updating facts when picking the flowers}

Now that we have our condition set, we need to update our facts. Facts can be
updated either when nodes are visited, or when links are clicked. In our case,
when the node``Pick the violets'' is visited, we'll update the fact ``Picked
flowers'' to \textit{true}, and update the fact ``The Flowers'' to hold the name
of the flowers that were picked.

\begin{enumerate}
  \item If you aren't already, edit the ``Add Anywhere Link'' rule in the
  node rule for node ``Pick the violets'' again.
  \item Add an ``update fact'' action.
  \item Set the action's fact type to ``True/False''. 
  \item Now choose the fact \textit{Picked flowers} and set the value to
  \textit{True}. This means that when the node ``Pick the violets'' is
  visited, the fact \textit{Picked flowers} is set to \textit{true}, ie. Red
  has picked the flowers.
  \item Add another ``update fact'' action, and select \textit{Text} as the
  type.
  \item Now choose the fact \textit{The flowers}. You should see a text entry
  field appear to the right. Type in the text ``violets''.
\end{enumerate}

\begin{figure}[h]
  \centering
  \includegraphics[width=10cm]{images/hypedyn-tutorial-3-figure-11}
  \caption{\textit{Updating facts for ``Pick the Violets''.}}
  \label{fig:tut3:picktheviolets2}
\end{figure}

You should have actions as in Figure \ref{fig:tut3:picktheviolets2}. Note that
in this image the original ``Enable links to this node from anywhere'' action
is scrolled off the top.

Now do exactly the same thing for the node ``Pick the geraniums'', except in
the last step, type in the text ``geraniums'' instead of ``violets''.

\subsection{Updating facts when in the forest grove}

At this point we are missing one key step - we don't ever update the fact ``In
the forest grove''. We want to use this to keep track of when the reader can have
Red pick the flowers. To do this, we need to set the fact to \textit{true} when
Red enters the forest, ie. when the reader visits the node ``Explore the
forest'', and set it to \textit{false} whenever any of the nodes which represent
leaving the forest grove are visited. These nodes are ``Go directly to Grandma's
house'', ``Head home'', and ``Go deeper into the forest''.

\begin{enumerate}
  \item Edit the node ``Explore the forest''.
  \item Edit the ``Add Anywhere Link'' rule for the node's node rule, and add
  an ``update fact'' action to set the fact ``In the forest grove'' to
  \textit{true}.
  \item Now do the same for nodes ``Go directly to Grandma's house'', ``Head
  home'', and ``Go deeper into the forest'', but instead of setting the fact to
  \textit{true}, set it to \textit{false}.
\end{enumerate}

We now have the ``In the forest grove'' fact set up to track whether or not Red
is in the forest grove. Note that this would be \textit{very} difficult to
accomplish using conditions on whether nodes where visited or not. (Try it!)

\subsection{Using text facts with conditional text}

The one thing remaining to be done is to update the text where the flowers are
mentioned, so that the specific type of flowers that Red picked are shown.
There are two places where this can be done: ``Pass the basket to Grandma'',
and ``Talk to young man''. We'll to the first, and the second is left as an
exercise.

\begin{enumerate}
  \item Edit the node ``Pass the basket to Grandma''.
  \item Select the text ``flowers'', and create a new link named ``flowers''.
  \item Add a new rule to the link, and edit the rule.
  \item Name the rule ``display the flowers''.
  \item Add an ``update text using'' action.
  \item Now change the type of text from ``alternative text'' to ``text
  fact'', and choose the fact \textit{The flowers}.

This means that the link text will be replaced by whatever value for the
fact ``The flowers'' we have set it to - either ``violets'' or ``geraniums'',
depending on which node the reader visited. 

\item We also need to make sure this replacement is only done if the flowers have
been picked. Add a condition, and select type \textit{Fact}. Choose the fact
\textit{Picked flowers}, and the value \textit{True}.
  This means that the default text will be shown if the flowers have \textit{not}
  been picked, and the value of the fact \textit{The flowers} will be shown if
  the flowers \textit{have} been picked.
\end{enumerate}

The link for the alternative text should look something like Figure
\ref{fig:tut3:conditional_text}.

\begin{figure}[h]
  \centering
  \includegraphics[width=10cm]{images/hypedyn-tutorial-3-figure-12}
  \caption{\textit{Substituting alternative text with a fact.}}
  \label{fig:tut3:conditional_text}
\end{figure}

Try running the story and picking the flowers. Notice that for either of the
flowers that are picked, the correct name is substituted in the node ``Pass the
basket to Grandma''. Add the same alternative text to ``Talk to young man''.

Note that, as with the other examples where we have used facts, we could have
done the same thing with regular node conditions and alternative text. However,
it would have required several links in the ``Pass the basket to Grandma'' node.
In addition, we can easily add a third flower, and no changes need to be made to
either of the nodes where the text is substituted. This allows for much more
systematic use of procedural change than the alternative text mechanism
introduced in tutorial 2.

\section{Next steps}

We have created a simple story using the ``sculptural'' approach to hypertext,
where all nodes implicitly have links between them, and the creation of the
story structure consists of using conditions to gradually restrict which nodes
are and are not available to the reader. We have also used facts to keep track
of what the reader has done, and to procedurally change the text in nodes. The
completed version of this story can be found in the file \textsc{LRRH4.dyn}.

There are several things that you could try to enhance the story. For example,
you could allow the reader to go back from ``Go deeper into the forest'' to pick
the flowers. This can easily be done by making use of the ``In the forest grove''
fact. Try adding a third type of flower to be picked. You might also want to go
back and use facts instead of node conditions to simplify the conditions we
placed on the anywhere nodes in the first part of this tutorial.

\section{Conclusion}

In this tutorial, we have created a simple ``sculptural'' hypertext fiction. By
creating a collection of story fragments, and then specifying a set of
conditions for when these fragments can be seen, we have taken a very different
approach to writing a hypertext story than was seen in the earlier versions of
HypeDyn. In addition, by using \textit{facts} to keep track of what the reader
has done, either as true or false conditions, or as text, we now have a much
more flexible way of changing the behaviour of the system based on the reader's
actions. This provides much more powerful possibilities in terms of
procedural change and interactivity.

% \section{Final notes}
% 
% As mentioned above, the HypeDyn hypertext editor is still a work-in-progress.
% Please let me know if you encounter any problems. Also, please post your
% questions about using the tool to the IVLE forum.

  \bibliographystyle{apalike}
  \bibliography{../../../Research/references/papers/thesis}

\end{document}
